% Preamble for documents
%
% Abhinav Sinha, sinha45@purdue.edu
% Created on Feb. 23, 2019
\usepackage{geometry}
\geometry{letterpaper, textheight = 680pt, tmargin = 80pt,
					left = 72pt, footskip = 24pt, headsep = 18pt, headheight = 14pt}
\usepackage{lmodern}
\usepackage{setspace}
\usepackage{textcase}
\usepackage{enumitem}
\usepackage{amsthm, amssymb, amsmath, mathtools, xfrac}
\usepackage{listings}
\lstset{
    columns = fullflexible,
    mathescape
}

\usepackage{hyperref}
\usepackage{cleveref}

% I want fancy section symbol
\crefname{section}{\S}{\S\S}
\Crefname{section}{\S}{\S\S}

\urlstyle{same}
\ifx\paragraph\undefined\else
\let\oldparagraph\paragraph
\renewcommand{\paragraph}[1]{\oldparagraph{#1}\mbox{}}
\fi
\ifx\subparagraph\undefined\else
\let\oldsubparagraph\subparagraph
\renewcommand{\subparagraph}[1]{\oldsubparagraph{#1}\mbox{}}
\fi

% commands
\newcommand{\problem}[1]{%
	\begin{flushleft}
		\textbf{\large Problem #1}
	\end{flushleft}
	\par
}

\newcommand{\ppart}[1]{%
	\textbf{Part #1}
	\par
}

\newcommand{\todo}{%
    -- TODO --
}

% math symbols
\renewcommand{\qedsymbol}{\(\blacksquare\)}
\DeclarePairedDelimiter{\abs}{\lvert}{\rvert}

\newcommand{\CC}{\mathbb{C}}
\newcommand{\KK}{\mathbb{K}}
\newcommand{\NN}{\mathbb{N}}
\newcommand{\QQ}{\mathbb{Q}}
\newcommand{\RR}{\mathbb{R}}
\newcommand{\ZZ}{\mathbb{Z}}

\newcommand{\Lagr}{\mathcal{L}}

\newcommand{\bigO}{\ensuremath{\mathcal{O}}}

\newcommand{\kron}{\otimes}
\DeclareMathOperator{\diag}{diag}
\DeclareMathOperator{\trace}{trace}
\DeclareMathOperator{\tvec}{vec}
\DeclareMathOperator{\rank}{rank}
\DeclareMathOperator{\tspan}{span}
\DeclareMathOperator*{\minimize}{minimize}
\DeclareMathOperator*{\maximize}{maximize}
\DeclareMathOperator{\subjectto}{subject\ to}

\newcommand{\mat}[1]{\boldsymbol{#1}}
\renewcommand{\vec}[1]{\boldsymbol{\mathrm{#1}}}
\newcommand{\vecalt}[1]{\boldsymbol{#1}}

\newcommand{\conj}[1]{\overline{#1}}

\newcommand{\normof}[1]{\|#1\|}
\newcommand{\onormof}[2]{\|#1\|_{#2}}

\newcommand{\MIN}[2]{\begin{array}{ll} \displaystyle \minimize_{#1} & {#2} \end{array}}
\newcommand{\MINone}[3]{\begin{array}{ll} \displaystyle \minimize_{#1} & {#2} \\ \subjectto & {#3} \end{array}}
\newcommand{\OPTone}{\MINone}
\newcommand{\MINthree}[5]{\begin{array}{ll} \displaystyle \minimize_{#1} & {#2} \\ \subjectto & {#3} \\ & {#4} \\ & {#5} \end{array}}

\newcommand{\MAX}[2]{\begin{array}{ll} \displaystyle \maximize_{#1} & {#2} \end{array}}
\newcommand{\MAXone}[3]{\begin{array}{ll} \displaystyle \maximize_{#1} & {#2} \\ \subjectto & {#3} \end{array}}


\newcommand{\itr}[2]{#1^{(#2)}}
\newcommand{\itn}[1]{^{(#1)}}

\newcommand{\prob}{\mathbb{P}}
\newcommand{\probof}[1]{\prob\left\{ #1 \right\}}

\newcommand{\pmat}[1]{\begin{pmatrix} #1 \end{pmatrix}}
\newcommand{\bmat}[1]{\begin{bmatrix} #1 \end{bmatrix}}
\newcommand{\spmat}[1]{\left(\begin{smallmatrix} #1 \end{smallmatrix}\right)}
\newcommand{\sbmat}[1]{\left[\begin{smallmatrix} #1 \end{smallmatrix}\right]}

\providecommand{\eye}{\mat{I}}
\providecommand{\mA}{\ensuremath{\mat{A}}}
\providecommand{\mB}{\ensuremath{\mat{B}}}
\providecommand{\mC}{\ensuremath{\mat{C}}}
\providecommand{\mD}{\ensuremath{\mat{D}}}
\providecommand{\mE}{\ensuremath{\mat{E}}}
\providecommand{\mF}{\ensuremath{\mat{F}}}
\providecommand{\mG}{\ensuremath{\mat{G}}}
\providecommand{\mH}{\ensuremath{\mat{H}}}
\providecommand{\mI}{\ensuremath{\mat{I}}}
\providecommand{\mJ}{\ensuremath{\mat{J}}}
\providecommand{\mK}{\ensuremath{\mat{K}}}
\providecommand{\mL}{\ensuremath{\mat{L}}}
\providecommand{\mM}{\ensuremath{\mat{M}}}
\providecommand{\mN}{\ensuremath{\mat{N}}}
\providecommand{\mO}{\ensuremath{\mat{O}}}
\providecommand{\mP}{\ensuremath{\mat{P}}}
\providecommand{\mQ}{\ensuremath{\mat{Q}}}
\providecommand{\mR}{\ensuremath{\mat{R}}}
\providecommand{\mS}{\ensuremath{\mat{S}}}
\providecommand{\mT}{\ensuremath{\mat{T}}}
\providecommand{\mU}{\ensuremath{\mat{U}}}
\providecommand{\mV}{\ensuremath{\mat{V}}}
\providecommand{\mW}{\ensuremath{\mat{W}}}
\providecommand{\mX}{\ensuremath{\mat{X}}}
\providecommand{\mY}{\ensuremath{\mat{Y}}}
\providecommand{\mZ}{\ensuremath{\mat{Z}}}
\providecommand{\mLambda}{\ensuremath{\mat{\Lambda}}}
\providecommand{\mSigma}{\ensuremath{\mat{\Sigma}}}
\providecommand{\mPbar}{\bar{\mP}}

\providecommand{\ones}{\vec{e}}
\providecommand{\va}{\ensuremath{\vec{a}}}
\providecommand{\vb}{\ensuremath{\vec{b}}}
\providecommand{\vc}{\ensuremath{\vec{c}}}
\providecommand{\vd}{\ensuremath{\vec{d}}}
\providecommand{\ve}{\ensuremath{\vec{e}}}
\providecommand{\vf}{\ensuremath{\vec{f}}}
\providecommand{\vg}{\ensuremath{\vec{g}}}
\providecommand{\vh}{\ensuremath{\vec{h}}}
\providecommand{\vi}{\ensuremath{\vec{i}}}
\providecommand{\vj}{\ensuremath{\vec{j}}}
\providecommand{\vk}{\ensuremath{\vec{k}}}
\providecommand{\vl}{\ensuremath{\vec{l}}}
\providecommand{\vm}{\ensuremath{\vec{l}}}
\providecommand{\vn}{\ensuremath{\vec{n}}}
\providecommand{\vo}{\ensuremath{\vec{o}}}
\providecommand{\vp}{\ensuremath{\vec{p}}}
\providecommand{\vq}{\ensuremath{\vec{q}}}
\providecommand{\vr}{\ensuremath{\vec{r}}}
\providecommand{\vs}{\ensuremath{\vec{s}}}
\providecommand{\vt}{\ensuremath{\vec{t}}}
\providecommand{\vu}{\ensuremath{\vec{u}}}
\providecommand{\vv}{\ensuremath{\vec{v}}}
\providecommand{\vw}{\ensuremath{\vec{w}}}
\providecommand{\vx}{\ensuremath{\vec{x}}}
\providecommand{\vy}{\ensuremath{\vec{y}}}
\providecommand{\vz}{\ensuremath{\vec{z}}}
\providecommand{\vxi}{\ensuremath{\vecalt{\xi}}}
\providecommand{\vpi}{\ensuremath{\vecalt{\pi}}}

\providecommand{\vlambda}{\ensuremath{\vecalt{\lambda}}}

% \DeclareMathOperator{\sinh}{sinh} 	already exists
% \DeclareMathOperator{\cosh}{cosh} 	already exists
% \DeclareMathOperator{\tanh}{tanh} 	already exists
\DeclareMathOperator{\csch}{csch}
\DeclareMathOperator{\sech}{sech}
% \DeclareMathOperator{\coth}{coth} 	already exists


\makeatletter
% course number (optional)
\newcommand{\cnum}[1]{\def\@cnum{#1}}
% subject (required)
\newcommand{\subject}[1]{\def\@subject{#1}}

\usepackage{fancyhdr}
	\pagestyle{fancy}
	\fancyhf{}
	\lhead{\scshape \@subject}
	\chead{\scshape \@title}
	\rhead{Pg.\,\thepage}
	
\def \maketitle {%
	\par
		\hrule height 0.75pt\vspace{0.5 ex}
		\hrule height 0.75pt\vspace{1 ex}
	\par\noindent
	\begin{minipage}{0.5 \textwidth}
		\MakeTextUppercase{\textbf{\@title}} \\
		\ifx \@cnum \undefined
			\scshape \@subject
		\else
			 \scshape \@cnum $\text{ }\cdot$ \@subject
		\fi
	\end{minipage}
	\begin{minipage}{0.495 \textwidth}
	\raggedleft
		\scshape
		\@author \\
		\@date
	\end{minipage}
	\par\vspace{1 ex}
		\hrule height 0.75pt\vspace{0.5 ex}
		\hrule height 1pt \vspace{2 ex}
	\par
}
\makeatother

% default fields
\title{Lecture Notes}
\author{Abhinav Sinha}
\date{\today}
\subject{\lbrack Subject\rbrack}

% clear first page header
\AtBeginDocument{%
	\thispagestyle{empty}
	% auto generate title?
	% \maketitle
}
